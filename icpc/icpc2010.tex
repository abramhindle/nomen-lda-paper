%!TEX TS-program = xelatex
%!TEX encoding = UTF-8 Unicode
\documentclass[10pt, conference, compsocconf]{IEEEtran}
\usepackage{fontspec}
\usepackage{xunicode}
\usepackage{xltxtra}
\usepackage{booktabs}
\defaultfontfeatures{Mapping=tex-text}
\setmainfont{Times New Roman}
%\usepackage{amsthm}
\usepackage{color}
\newcommand{\ppe}[2]{$#1 \xrightarrow{++} #2$}
\newcommand{\pe}[2]{$#1 \xrightarrow{+} #2$}
\newcommand{\nne}[2]{$#1 \xrightarrow{--} #2$}
\newcommand{\nege}[2]{$#1 \xrightarrow{-} #2$}
\newcommand{\ande}[2]{$#1 \xrightarrow{AND} #2$}
\newcommand{\ore}[2]{$#1 \xrightarrow{OR} #2$}
\newcommand{\inode}[2]{$#1 \xrightarrow{I} #2$}
\newcommand{\cnode}[2]{$#1 \xrightarrow{C} #2$}
\newcommand{\ab}[1]{}

\newtheorem{mydef}{definition}
\usepackage{graphicx}
\newcommand{\slbl}[1]{\textbf{\textsf{#1}}}
\usepackage{multirow}%,multicolumn}
\usepackage{slashbox}
\usepackage{wrapfig}
\usepackage{booktabs, colortbl}
\usepackage{amsmath,amssymb}
\usepackage{algpseudocode}
%\usepackage[pdfstartview=FitH,xetex,bookmarksopenlevel=3,bookmarks=true]{hyperref} %pdfpagescrop={92 112 523 778},
\usepackage{algorithm}%,algorithmic-fix}
\begin{document}
 
\author{
\IEEEauthorblockN{Abram Hindle}
\IEEEauthorblockA{Department of Computer Science\\University of Waterloo\\ ahindle@uwaterloo.ca\\}
\and
\IEEEauthorblockN{Neil A. Ernst}
\IEEEauthorblockA{Department of Computer Science\\ University of Toronto\\ jm,nernst@cs.toronto.edu}

}


\title{What's in a name? On the (automated) topic naming of software maintenance activities}

\maketitle
\thispagestyle{empty}

\begin{abstract}
  In automated mining of software repositories, most approaches, such as topic modeling and concept location, focus on applying standard machine learning algorithms to corpora. This approach does not make any use of domain-specific information to better understand results. While too much specificity can produce non-generalizable results, too little produces broad learners that don't provide much useful detail.%There has not been any   investigation into the latent topics these classifiers represent. 
In this paper, we investigate the latent topics classifier produce, and examine  whether such topics can be
  generalized across software products. We focus on software quality
  as a potential generalization, since there is some shared belief
  that these qualities apply broadly across software products. We
  wanted to name topics based on a static ontology of non-functional
  requirements and software qualities. %Our overarching goal is to use   individual topic models to create a list of terms that can be shared   across software projects. 
In essence, this would be adding some form
  of context to general-purpose ontologies such as Wordnet. In this
  paper, we discuss how to use domain knowledge about software
  development in order to better annotate topics of development that
  are extracted from the change log messages in configuration
  management systems such as CVS. Since our taxonomy is global, our results allow us to compare the relative importance of particular qualities between projects.
\end{abstract}



\section{Introduction}
In this paper we `name' topic models generated with a topic modeling technique, Latent Dirichlet Allocation. We also show that these topics can be generalized across products.

Traditionally topic modeling has required manual annotation to derive domain-relevant information. This paper introduces \emph{named topic models}, topic models which are automatically named using a cross-project taxonomy. We start with an unsupervised topic learning algorithm, then manually annotate the topics using the ontology. From the annotated topics, we apply supervised classifiers to find labels for the remaining topics. 

Typically what is distinct is the software's particular functionality -- whether it offers email composition, prepares tax returns, and so on. What is common, however, is the non-functional, or quality requirements, that the software focuses on. Our approach makes use of software quality taxonomies in order to compare different projects on similar concepts. For example, how is the Security quality treated over time by product A versus product B? Does product A care about it more as it matures? Does the quality occur more often in software commits?

We first introduce some important terminology for our work. We then describe our methodology, including our datasets, then highlight our results. We conclude with a look at related work and possible improvements.

\section{Background}
We provide a brief overview of software repository mining and information retrieval.
\subsection{Definitions}
A \emph{message} is a block of text written by a developer. In this
paper, messages will be the CVS and BitKeeper commit log comments made
when the user commits changes to files in a repository. A \emph{word
  distribution} is the summary of a message by word count. Each word
distribution will be over the words in all messages. However, most
words will not appear in each message. A word distribution is effectively
a word count divided by the message size. A \emph{topic} is a word
distribution, i.e., a set of words that form a word distribution that is
unique and independent within the set of documents in our total
corpus. One could think of a topic as a distribution of the centroid
of a group of messages. In this paper we often summarize topics by the
top 10 most frequent words of their word distribution.  A \emph{trend}
is one or more similar topics that recur over time.  Trends are
particularly important, as they indicate long-lived and recurring
topics that may provide key insights into the development of the
project.

\emph{ROC} values reflect a score, similar to school letter-grades, reflecting how well a particular learner performed for the given data. A ROC result of 0.5 would be equivalent to a random learner (randomly classifying data). ROC maps to the more familiar concepts of precision/sensitivity and recall/specificity: it plots the true positive rate (sensitivity) vs. the false positive rate (1 - specificity). A perfect learner has a value of 1.0, reflecting perfect recall and precision.

\subsection{Topic models}
Topic analysis uses tools such as Latent Dirichlet Allocation (LDA)~\cite{Blei2003} and Latent Semantic Indexing (LSI) to extract
independent word distributions (topics) from
	documents (commit log comments). %~\cite{marcus04wcre,Poshyvanyk2007,lukins08wcre,Linstead2007}
Ideally these extracted topics 
correlate with actual development topics that the developers discussed
during the development of the software project. For example, our technique might identify a collection of words: `list next has iterator add', which we then identify (manually) as concerning the topics of collections and the Iterator pattern.
In topic analysis a single document, such as a commit message, can be related to multiple topics. Representing documents as a mixture of topics maps well to commits to source code repositories, which often have more than one purpose.  A topic
represents both a word distribution and a group of commit log comments
that are related to each other by their content.  In this paper a topic
is a set of tokens extracted from commit messages found within a
project's source control system (SCS).

In this paper we use LDA to identify topics.  LDA works by creating topic models for each period. A topic model is a statistical summary of the important subjects for that dataset.  The input to the algorithm is a set of documents, with contents treated as a bag of words, and a parameter to set the number of topics to generate. The algorithm returns a word-document matrix relating the occurrence frequency of a 
 The idea is similar to other IR approaches like TF/IDF (term frequency/inverse document frequency), which find uncommon words that aren't found in other wordlists. These approaches are unsupervised in that they rely solely on the source data with no human intervention.

One issue that arises with use of unsupervised techniques is how to label the topics. While the topic models themselves are generated automatically, what to make of them is less clear. For example, in our previous work~\cite{Hindle09ICSM}, and in \cite{Baldi2008}, topics are named manually: human experts read the highest-frequency members of a topic and assign a keyword accordingly. E.g., for the word list ``listener change remove add fire", Baldi et al. assign the keyword `event-handling'. The labels are reasonable enough, but still require some expert in the field to determine them. Our technique is automated, because we match keywords from Wordnet to words in the topic model. 

\subsection{Supervised learning}
While unsupervised techniques (LSI and LDA are both unsupervised) are appealing in their lack of human intervention, supervised learners have the advantage of domain knowledge. This typically results in improved results. In supervised learning, the data is divided into slices. One slice is manually annotated by the domain expert

 
\section{Methodology}
 % To gather the data, we first extract the commit log comments from a
% project's SCS repository. We filter out stop words and produce word distributions from these messages. These distributions are bucketed
% into 30-day non-overlapping windows. Each window is subject to LDA analysis. %Figure \ref{fig:commits}
% depicts the general process for processing and analyzing the commit messages.


We wanted to compare two projects with open, accessible source code in similar functional areas. This allows us to partially control for differences in functional requirements. We selected MySQL and MaxDB, open-source, near-commercial quality database software. MaxDB started in the late 1970s as a research project, and was later acquired by SAP. As of version 7.5, released April 2007, the project has 940 thousand lines of C source code\footnote{generated using David A. Wheeler's `SLOCCount'.}. There are approximately ... developers working on it. MySQL 3.23, released in early 2001, contains 320 thousand lines of C and C++ source code. The MySQL project started in 1994.

From each project, we used source code commit data (the message attached to the commit). We leveraged the data we gathered in \cite{Hindle09ICSM} for this work. A typical commit message is: ``history annotate diffs bug fixed (if mysql_real_connect() failed there were two pointers to malloc'ed strings, with memory corruption on free(), of course) ". We extracted these messages and indexed them by creation time. 

From that dataset, we created an XML file which grouped commits into temporal periods, each approximately 4 weeks wide. Given a text file full of commits, indexed by date and time, we then ran an LDA modeler on each period, to generate a list of topics per period. E.g., in Period 1, the most important topic might have been ``security", with ``sql'' second, down on to less relevant topics. We arbitrarily set the number of topics to generate to 20, based on past experiences.

	Once we have a list of topics for each period, we compared them with lists of keywords to assess to what extent our topics fit word lists. Intuitively, we would like to `name' our topic models for each period. In this way, we can see how certain words -- subjects -- occur more or less often with each period. We used several different word lists. One we tried was 9126, the ISO quality model~\cite{iso9126}. It describes six high-level quality requirements (listed in Table \ref{}). There is some debate about the significance and importance of the terms in this model. However, it is ``an international standard and thus provides an internationally accepted terminology for software quality \cite[p. 58]{Bøegh2008},'' which is sufficient for the purposes of this research. %exp3

	We then extended our word lists with related words using Wordnet.
	
	 If words in the quality model were also found in a period, we automatically assigned that period to that concept (e.g., security, performance, reliability). 

To check our outcomes, we manually annotated each period with the same quality model. In other words, we looked at each period, and tried to assess what the LDA analysis output told us was the topic for that period.

We then compared our automated analysis using signifier matching (greps) to the manual tagging to get an error rate.

To compare our LDA approach, we used a suite of supervised classifiers from WEKA (e.g., support vector machines and Bayesian nets). These learners, described elsewhere, may also be used to find topics. We did a 19/20 analysis (using 1 period to train the learner on the tags, and the other 19 to attempt to learn) and got the following error rates. 

We present a ROC curve in Fig. \ref{}. The chart shows each learner measured in terms of precision and recall. In general, we prefer techniques which are closer to the upper-left corner (maximizing precision and maximizing recall). 

% - Extract topics from changes
% - Annotate topics of multiple projects
% -- mysql
% -- maxdb
% - Try simple word matching
% - Try simple word match of wordnet words (related words)
% - Try machine learning
% - Make a simple wordnet?
% - Try it


\section{Observations and evaluation}

\section{Related work}
Part of our effort with this project is to understand the qualitative and intentional aspects of requirements in software evolution, a notion we first discussed in \cite{ernst07icsm}. That idea is derived from, in part, work on narratives of software systems shown in academic work like \cite{anton01}, or more general-purpose works like \cite{waldo93}.

The idea of extracting higher-level `concerns' (also known as `concepts', `aspects' or `requirements') has been attacked in two ways.

Cleland-Huang and her colleagues published work on mining requirements documents for non-functional requirements (quality requirements) in \cite{Cleland-Huang2006}. One approach they tried was similar to this one, with keywords mined from NFR catalogues found in~\cite{chung99}. They managed recall of 80\% with precision of 57\% for the Security NFR, but could not find a reliable source of keywords for other NFRs. Instead, they developed a supervised classifier by using human experts to identify an NFR training set. There are several reasons we did not follow this route. One, we believe we have a more comprehensive set of terms due to the taxonomy we chose. Secondly, we wanted to compare across projects. Their technique was not compared across different projects and the applicability of the training set to different corpora is unclear. A common taxonomy allows us to make inter-project comparison (subject to the assumption that all projects conceive of these terms in the same way). Thirdly, while the objective of Cleland-Huang's study was to identify new NFRs (for system development) our study assumes these NFRs are latent in the textual documents of the project. 
Finally, the source text we use is less structured than their requirements documents.

The other approach is to start with code repositories, and try to extract concerns from there. In \cite{marcus04wcre}, the authors describe their use of Latent Semantic Indexing to identify commonly occurring concerns for software maintenance. Some results were interesting, but their precision was quite low.  (such as 
ConcernLines~\cite{treude09cl} shows tag occurrence using color intensity. They mined change request tags (such as `milestone 3') and used these to make evolutionary analyses of a single product. The presence of a well-maintained set of tags is obviously essential to the success of this technique.
%We maintain that these results require more substantive manipulation before being reported to, for example, management. Like any large-scale data analysis project, it requires skill to derive the implications. Hooking the backend results directly to a reporting tool is unlikely to produce useful results.

% There has been an explosion of interest in mining OSS repositories. Godfrey and Tu \cite{godfrey00} were among the first to use this data to assess software evolution. They discussed how well the growth of Linux was predicted by laws that Lehman and colleagues \cite{lehman85} proposed. More recently, Herraiz et al. \cite{herraiz07icsm} predicted OSS growth using time series analysis. Like us, they refer to release windows as a useful unit of analysis. 

Mens et al. \cite{mens08icsm} conducted an empirical study of Eclipse, the OSS code editor, to verify the claims of Lehman~\cite{lehman97sms}. They concerned themselves with source code only, and found Law Seven, ``Declining Quality'', to be too difficult to assess: ``[we lacked an] appropriate measurement of the evolving quality of the system as perceived by the users \cite[p. 388]{mens08icsm}''. This paper examines the notions of quality in terms of a consistent ontology, as Mens et al. call for in their conclusions.

Massey~\cite{massey02icse} and Scacchi (\cite{scacchi02,scacchi05b}) looked at the topic of requirements in open-source software. Their work discusses the source of the requirements and how they are used in the development process. German~\cite{german03gnome} looked at GNOME specifically, and listed several sources for requirements: leader interest, mimicry, brainstorming, and prototypes. None of this work  addressed quality requirements in OSS, nor did it examine requirements trends.

Hindle et al. \cite{Hindle2007} examined release patterns in OSS. They showed that there is a difference between projects regarding maintenance techniques. This supports our result that software qualities are not discussed with the same frequency across projects.

Most of this related research examined project source code. Software mining of other project corpora is less common. Similarities exist with approaches that begin with structured taxonomies, as with the Hismo software ontology \cite{girba06}. Rigby and Hassan \cite{rigby07msr} used a general purpose taxonomy to classify developer email according to temperament.

\section{Conclusion}
% ``Can a domain specific wordnet for
%    software development provide better accuracy for labelling and other
%    lexical related tasks than wordnet and other machine learning
%    techniques''.


\bibliographystyle{IEEEtran}
\bibliography{icpc}

\end{document}


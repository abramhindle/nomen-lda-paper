\documentclass{report}
\usepackage{graphicx,amsmath,comment,fullpage}
\newcommand{\igH}[1]{\includegraphics[height=.9\textheight]{#1}}
\newcommand{\igW}[1]{\includegraphics[width=.9\textwidth]{#1}}
\newcommand{\igWh}[1]{\includegraphics[width=.7\textwidth]{#1}}
\newcommand{\igWhalf}[1]{\includegraphics[width=.45\textwidth]{#1}}
\newcommand{\lda}{Latent Dirichlet Allocation}


\author{Abram Hindle\\
ahindle@cs.uwaterloo.ca\\
Software Architecture Group (SWAG)\\
Supervised by\\
Micheal Godfrey and Ric Holt
}
\title{Lab 0006: Latent Dirichlet Allocation Applied to Revisions}

\usepackage{rotating}   
\begin{document}

\maketitle
\begin{abstract}

  In this lab report we seek to explore how topics shift over time in
  the source control repository of a project. We analyze commit
  comments, and even source code per revision per each time window and
  extract the topics prevelant in that window. We expect that topic
  sets will change over time and the change in topic sets indicate a
  change in the focus of development. We suppose this could help us to
  segment the revisions on the time line to discover parts of an
  iteration. For topic extraction we use \lda (LDA).

\end{abstract}

\section{Introduction}

%?


% 

% In the presence of fine grained changes and detailed changelogs

% Developers topics the focus of develop

% many changelogs how to sort by change type

% interleaved changes relate similar changes

What happened in the previous iteration? What were the developers
working on? What was the developer in the next cubicle working on?
What were the common topics and issues dealt with in the current
release of the software? What were the topics of the previous release?

We propose to extend common concept analysis techniques by executing
them over windows of documents over time. We hope by looking at topic
clusters over time we can I identify continuous effort on a select
group of topics.

Topic analysis of fine-grained revisions is important as often
revisions belonging to different topics are interleaved or even mixed
inside of commits.  Sometimes a single revision to a file contains
multiple bug fixes.  Using powerful topic analysis like Latent
Semantic Indexing (LSI) or Latent Dirichlet Allocation (LDA) one can
seperate these mixed signals into coherent topics. In this case a
topic is a set of tokens that are related to each other.

% We suppose this could help us to segment the revisions on the time
% line to discover parts of an iteration.

Topic analysis is useful because it can help us partition a project's
time-line into sub-iterations. By breaking apart an iteration we might
be able to extract the underlying software development process from
these artifacts.

In this paper we seek to explore how topics shift over time in
the source control repository of a project. We analyze commit
comments, and even source code per revision per each time window and
extract the topics in that time window. We expect that topic sets
will change over time and the change in topic sets indicate a change
in the focus of development. 

\section{Previous Work}

\subsection{Intro to LDA}

XXX explain LDA \cite{944937}

\subsection{LDA Work}

Lukins, Kraft and Etzkorn~\cite{lukins2008} use LDA to help query for
finding related documents during bug localization tasks. They used LDA
to build a topic model and then query against it using sample
documents, the query would return relevant documentation.

LSI is related to LDA and has been used to identify topics in software
artifacts~\cite{1421013,1374321,10.1109/ICPC.2007.13,10.1109/ICPC.2006.17}.

Semantic Clustering \cite{1698774,1566153}

Linstead et al has proposed an author source code model using
LDA.\cite{10.1109/MSR.2007.20,NIPS2007637,1321709} 

Grant et al. have used an alternative technique to LSI and LDA, they
used Independant Component Analysis to seperate topic signals from
software code.

Our technique differs in the sense we apply LDA temporally and we
apply to change comments.

%      1. [ ] LDA itself
%      2. [ ] Maletic stuff
%      3. [ ] WCRE: Scott/Jim
%      4. [ ] WCRE: Bug LDA stuff
%      5. [ ] MSR: Challenge work
%      6. [ ] FCA work
%4. [ ] Methodology [0/9]



\section{Methodology}

\subsection{Extract Repos}
%   2. [ ] Extract Documents
\subsection{Extract Documents}
%   3. [ ] Windowed Sets
\subsection{Windowed Sets}
%   4. [ ] Cluster Windowed Set w/ LDA
\subsection{Cluster Windowed Set w/ LDA}
%   5. [ ] Extract Clusters
\subsection{Extract Clusters}
%   6. [ ] Cluster Similarity [0/1]
\subsection{Cluster Similarity}
%      1. [ ] need the tool
%             grab from the cluster analysis
%   7. [ ] Continuous Cluster Analysis
\subsection{Continuous Cluster Analysis}
%   8. [ ] Topic Smear
\subsection{Topic Smear}
%   9. [ ] Visualization [0/1]
\subsection{Visualization}
%       1. [ ] need the visualizer



\subsection{Datasets and Tools}

Extractors:
\begin{itemize}
\item CVSSuck - CVS Suck mirrors RCS files from a CVS repository. 
\item softChange - extract CVS facts to a PostgreSQL database.
\item bt2csv - Convert BitKeeper repositories to facts in CSV
  databases.
\end{itemize}

Analysis Tools:
\begin{itemize}
\item Hiraldo-Grok - An OCaml based spin off of Grok used for
  answering queries.
\item Gnuplot - graph plotting package
\end{itemize}


%   1. [ ] LDA

%   2. [ ] Extractor

%   3.   Similarity

%   4.   Smearing

%   5.   Visualizer



\section{First Impression}

% 2007-10-12

Looking at 30 day non-overlapping windows of revisions for MySQL 3.23
(20 topics, top 10 words per topic) we found there were common words
across topic clusters such as diffs, annotate and history. There were
notable transitional topics such as in the first window the word
bitkeeper appears (probably when they adopted bitkeeper) yet in the
following windows there were no mentions of bitkeeper. ``RENAME'' also
only appeared once in the first window.

Often words are shared across topics, see table \ref{tab:portability}
for tracking topics about portability.

It is apparent that we have to compare these topics some how by
itemset or similarity.

\begin{table}
\centering
\begin{tabular}{|ccc|l|}
\hline
2000 &  Jul &  31 &    chmod \\
2000 &  Sep &  29 &    fixes benchmark logging windows \\
2000 &  Nov &  28 &    typo fix insert\_multi\_value \\
2001 &  Jan &  27 &    FIxes Innobase Cleanups auto-union \\
2001 &  Mar &  28 &    2 topics bugfix, logging , TEMPORARY,  \\
\hline
2001 &  Jul &  26 &    Update Allow TABLES LOCK [a] \\ 

2001 &  Aug &  25 &    Tables row version [a] \\
\hline
2001 &  Sep &  24 &    Update checksum merge \\
2001 &  Oct &  24 &    fixed fix \\
2001 &  Dec &  23 &    HPUX SCO fix \\
\hline
2002 &  Feb &  21 &    net buffer length  max\_allowed\_packet [b] \\
2002 &  Mar &  23 &    small buf fix [b]  \\
\hline
2002 &  May &  22 &    [popular] fix SCO OSF1 table\_name \\
2002 &  Nov &  18 &    HPUX11 compiler HP \\
\hline
2003 &  Feb &  16 &    Linux errno  [c] \\
2003 &  Mar &  18 &    alarm bookmark bug [c] \\
\hline
2003 &  Sep &  14 &    Auto logging merge windows distribution fix 64-bit 4.0 Cleanup \\
\hline
\end{tabular}
\caption{Tracking topics associated with the word portability, note some continuous blocks}
\label{tab:portability}
\end{table}

%LDA Skeleton
%1. [ ] Abstract 
%2. [ ] Introduction

%3. [ ] Previous Work [0/2]
%   2. [ ] LDA Work [0/6]
\section{Results}
%   1. Interesting smears 
%   2.   postgresql
\subsection{Postgresql}
%   3.   mysql
\subsection{Mysql}
%   4.   firebird
\subsection{Firebird}
%   5.   what's that other db?
\subsection{ what's that other db?}
%7.   Validation [0/1]
\subsection{Validation}
%   1. well i'm stumped
%   2. look at it?
%   3. investigate those revisions and check the proportion etc
%   4. exploratory
%   5.   work it out
%8.   Validity Threats [0/5]
\section{Validity Threats}
%   1.   validation
%   2.   LDA is questionable
%   3.   blackbox problem
%   4.   not in the token
%   5.   multiple X token
%9.    Future Work
\section{ Future Work}
%10.   Conclusions
\section{Conclusions}
%11.   Start file file:/home/abez/projects/lda-paper/lda-paper.tex


\bibliographystyle{alpha}
\bibliography{paper1}


\end{document}

\documentclass{article}

\usepackage{comment,fullpage}

\begin{document}
Dear EMSE Reviewers,

We would like to thank you for your thorough reviews. In our revised paper,
we have addressed all five  major revision points. In addition, we incorporated nearly all of the
minor reviewer comments and corrections. 
% one exception: the timeline figure is still fairly small.
%We have a giant TODO file that was 
% compiled from the very thorough four reviews our paper received, and
% we achieved, met, and integrated all of those comments, edits, and
% requirements into this revision of the paper.

We now explain how we have amended the paper to incorporate the major revisions requested.

\subsubsection*{1. The reviewers would like to see discussion of the relationship of
 the current draft and the previous MSR 2011 paper and more detailed
 discussion of previous papers related to the current draft.}

In this new revision, within the introduction and the previous work we
addressed the provenance of this paper and clearly indicated what we
added to this paper to make it different and more thorough. We
addressed our better coverage of machine learning, our addition of the
PostgreSQL case study, our inter-rater reliability study, our author
analysis study, etc. 

We also fleshed out much of the previous work that both authors had
worked on in the past relevant to this paper, in order to provide more
context.

\subsubsection*{2. The reviewers have several requests for more details of the study
   design choices (e.g., choosing projects from the same application
   domain, using ROC instead of F-Measure as your primary measure) and
   some related discussion on the threats to validity.}

Throughout the paper we added more details, especially in terms of
methodology and what statistical methods we were using and how.

In the section ``Creating a Validation Corpus'' we added why we chose
ROC over F-Measure and described some of the weakness of F-Measure
based evaluation. We also motivated why we did provide both
measures. We also discussed some of these issues in threats to
validity.

Furthermore we addressed issues of low ROC and F-measure scores in
this section, in threats to validity, and in the conclusions.

We also addressed reviewer concerns about the use of a single domain.

Examples of further details include: we discussed the pre-processing
of changelog comments for LDA in better detail; we described our
clustering methodology clearer and discussed significance of many of
the comparisons; we provided some more explanation of LDA and the
topics it extracts.

\begin{comment}
         - Re: in the "Creating a Validation Corpus" we addressed why we
           used ROC over F-Measure: class imbalance leads to bias in
           F-Measure. And we cited relevant work that discussed this
           particular issue.            

         - Re: we explained some reasons for low ROC scores.

         - Re: in threats to validity, internal validity, we discussed skewed classes

     - Re: we responded on page 6 and in the threats to validity about
       the issues of a single domain

    - Re: discussed LDA word distributions

    - in the clustering section we described our methodology clearer
    - we better described LDA preprocessing
\end{comment}

\subsubsection*{
  3. Fix writing: The reviewers point out quite some inconsistent and
  unclear writing in various places of the paper.}

In an effort to clean up the paper and appease the reviewers, we
collated a big TODO list from the reviewer comments and ensured all of
the comments were addressed and all edits were executed. We feel that
cleaned up the paper and edited it very careful, with the help of the
reviewer comments of course.  As per reviewer requests we softened
some of the claims in light of low ROC and IRR scores when applicable.
We recognize there were a lot of edits suggested by the reviewers and
we did our due diligence and attended to all that we could.

\subsubsection*{4.  Reviewer 3 raises that the extremely low IRR (~0.1) threatens the
   validity of the results of these experiments, and its threats to
   validity should be made more prominent.}

Within this new revision, we addressed this issue in the inter-rater
reliability section and referenced this discussion in the threats to
validity and the conclusion. 

Furthermore we provided more evidence that our IRR results were mostly
better than noise, we did extensive simulations sampling our ratings
and then comparing those ratings against our ratings to find that the
majority of our labels received higher IRR scores than 96\% to 99.9\%
of the random simulations. This allows us to confidently show that our
labels were primarily better than random. But we recognize that the
average of the values is low and we address the threats this causes to
this work and what it might mean for practitioners and other
researchers when replicating this study or methodology.

We also addressed the reviewers concerns that the IRR scores would
affect our ROC and F-Measure scores. We addressed this in the IRR
section and the threats to validity.

\subsubsection*{5. Reviewer 4 raises the some observed results may simply be due to
 the differences among developer styles and not necessarily
 reflecting different types of work. }

In this new version, we addressed these issues ``Do different
developers work on different NFRs?''  and the threats to validity. We
also better explained the methodology behind comparing these authors
and the topics they are associated with.

In conclusion we addressed the reviews to fullest extent possible,
both in terms of major criticisms and minor criticisms. We feel that
this new revision is far more robust and show ease the concerns of the
reviewers.

Thanks you,

Abram, Neil, Michael and John



\end{document}

%%% Local Variables: 
%%% mode: latex
%%% TeX-master: t
%%% End: 

\documentclass{article}

\usepackage{comment,fullpage}

\begin{document}
Dear EMSE Reviewers,

\vspace{2mm}

We would like to thank you for your thorough reviews. In our revised paper,
we have addressed all five  major revision points. In addition, we incorporated nearly all of the
minor reviewer comments and corrections. 
% one exception: the timeline figure is still fairly small.
%We have a giant TODO file that was 
% compiled from the very thorough four reviews our paper received, and
% we achieved, met, and integrated all of those comments, edits, and
% requirements into this revision of the paper.
We now explain how we have amended the paper to incorporate the major revisions requested.

\subsubsection*{1. The reviewers would like to see discussion of the relationship of
 the current draft and the previous MSR 2011 paper and more detailed
 discussion of previous papers related to the current draft.}

In this new revision, we have expanded the introduction and the previous work sections to address
the provenance of this paper, and included a detailed description of the work added to 
this paper to extend the conference publication. In particular, we highlighted how we 
addressed our better coverage of machine learning, our addition of the
PostgreSQL case study, our inter-rater reliability study, and our author
analysis study.

We also expanded on the previous work that the authors had conducted 
 which was relevant to this paper, in order to provide more
context.

\subsubsection*{2. The reviewers have several requests for more details of the study
   design choices (e.g., choosing projects from the same application
   domain, using ROC instead of F-Measure as your primary measure) and
   some related discussion on the threats to validity.}

Throughout the paper we added explanations of 
methodology choices and choice of statistical methods.
In Section 3.1.2, ``Creating a Validation Corpus'', we added an explanation of our choice of 
ROC over F-Measure, referencing other researchers in describing some of the weakness of F-Measure
based evaluation. We now motivate why both measures were provided.
We also discussed some of these issues in Section 5.4, ``Threats to
Validity''.

Furthermore, we now address issues of low ROC and F-measure scores in
both 3.1.2 and 5.4, as well as in the conclusions. We added a discussion regarding concerns about the use of a single domain in section 3.1.

Examples of further amendments: we discussed the pre-processing
of changelog comments for LDA in better detail; we described our
clustering methodology more clearly, and discussed the statistical significance of
the comparisons; and we provided some more explanation of LDA and the
topics it extracts.

\subsubsection*{
  3. Fix writing: The reviewers point out quite some inconsistent and
  unclear writing in various places of the paper.}

We collated a big TODO list from the reviewer comments and ensured all of
the comments were addressed and all edits were executed. 
%We feel that cleaned up the paper and edited it very careful, with the help of the
%reviewer comments of course.  
As per reviewer requests we softened some of the claims in light of low ROC and IRR scores when applicable.
We recognize there were a lot of edits suggested by the reviewers and
we did our due diligence and attended to all that we could. We regret that making the figure showing timelines 
larger would have required three pages of figures. 
%We note that this figure is vector-based, and so zooms appropriately
%when read on a device. 

\subsubsection*{4.  Reviewer 3 raises that the extremely low IRR (~0.1) threatens the
   validity of the results of these experiments, and its threats to
   validity should be made more prominent.}

Within this new revision, we addressed this issue in a new section on inter-rater
reliability (5.3) and referenced this discussion in the threats to
validity (5.4) and the conclusion. 

Furthermore, we expanded the discussion with statistical evidence that our IRR results were mostly
better than noise. We did extensive simulations sampling our ratings,
and then comparing those ratings against our ratings. We found that the
majority of our labels received higher IRR scores than 96\% to 99.9\%
of the random simulations. This allows us to confidently state that our
labels were primarily better than random. But we recognize that the
average of the values is low, and Section 5.3 address the threats this causes to
this paper, and what it might mean for practitioners and other
researchers when replicating this study. Section 5.3 also addresses the reviewers concerns that the IRR scores would
affect our ROC and F-Measure scores. 

\subsubsection*{5. Reviewer 4 raises the [issue that] some observed results may simply be due to
 the differences among developer styles and not necessarily
 reflecting different types of work. }

We discuss this possibility in Section 4.3, ``Do different
developers work on different NFRs?'',  and in Section 5.3, the threats to validity. We
also better explained the methodology behind comparing these authors
and the topics they are associated with. 

We believe we have addressed the reviews to the fullest extent possible,
both in terms of major criticisms and minor criticisms. We feel that
this new revision is far more robust and should ease the concerns of the
reviewers.

\vspace{2mm}

Thank you,

\vspace{3mm}

Abram, Neil, Michael and John



\end{document}

%%% Local Variables: 
%%% mode: latex
%%% TeX-master: t
%%% End: 

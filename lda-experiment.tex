\documentclass{article}

\begin{document}

\section{Word lists}

Make a word list (a file) of related words and a label. Used wordnet
similarity (unfiltered) and concepts from \cite{5072519}.



\end{document}

%%% Local Variables: 
%%% mode: latex
%%% TeX-master: t
%%% End: 

@INPROCEEDINGS{5072519,
title={Towards an Ontology for Software Product Quality Attributes},
author={Kayed, A. and Hirzalla, N. and Samhan, A.A. and Alfayoumi, M.},
booktitle={Internet and Web Applications and Services, 2009. ICIW '09. Fourth International Conference on},
year={2009},
month={May},
volume={},
number={},
pages={200-204},
abstract={Recently, quality assurance concept has been developed increasingly to be included in many of our life existing fields; financial, industrial, trading, computing, etc. Software quality product attributes (SWQAs) have been created as a matter of applying the QA concept on the results of Web or desktop application development process, to fit the products with the organizational and global market standards and goals, and to provide it with a competitive advantage value. Web application or software product quality is composed of many attributes such as portability, usability, reliability, modularity. During the recent years, many researchers discussed and presented software attributes in their works which showed that till now there is a lack of consensus on the semantic of many of concepts and terminologies used in this field. Our work is focusing on studying software product quality attributes concepts and terminologies. We conduct several experiments to extract the main concepts for SWQAs. The results show that there is a number of concepts that are frequently used to describe these attributes. Summarizing and formalizing the semantic of the attributes into these concepts presents a common understanding and agreement on the semantic of SWQPAs which can be used by software engineers, researchers, practitioners, and stakeholders.},
keywords={quality assurance, software portability, software quality, software reliability, standardsWeb application development process, competitive advantage value, desktop application development process, global market standards, modularity, organizational market standards, portability, quality assurance, reliability, software product quality attributes, usability},
doi={10.1109/ICIW.2009.36},
ISSN={}, }


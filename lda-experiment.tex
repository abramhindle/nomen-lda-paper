\documentclass{article}

\begin{document}

% \section{Word lists}

% Make a word list (a file) of related words and a label. Used wordnet
% similarity (unfiltered) and concepts from \cite{5072519}.

% \subsection{Observations}

% We ran it and we found that wordnet for correctness was too
% generic. We ran it on MaxDB 7.500 with the words: integrity, security,
% interoperability, testability, maintainability, traceability,
% accuracy, modifiability, understandability, availability, modularity,
% usability, correctness, performance, verifiability, efficiency,
% portability, flexibility, and reliability.

% We had 500 topics extracted.





\section{Exp1: Word Lists on MaxDb 7.500 commit log comments}

We extracted the commit log comments from MaxDB 7.500 and per month extracted 20 topics using LDA. There are 25 months, thus a total of topic 500 possible topics, but there were a couple of months with no commits.

Make a word list (a file) of related words and a label. We used
wordnet similarity (unfiltered) on concepts extracted from
\cite{5072519}.  The concepts used included: integrity, security,
interoperability, testability, maintainability, traceability,
accuracy, modifiability, understandability, availability, modularity,
usability, correctness, performance, verifiability, efficiency,
portability, flexibility, and reliability.

We determined if a word list was similar to a topic if they shared 1
or more words in common.

\subsection{Observations}

We ran it and we found that wordnet generated word lists for
\emph{correctness} were too large and generic. Results are listed below
in the results section

\subsection{Future attempts:}

\begin{itemize}
\item  Make better manual word lists
\item  Take that ISO standard that Neil used
\item  Take that ISO Architecture standard that ric mentioned
\item  Different similarity measure
\item  Utilize wordnet distance
\item  Utilize LDA to query its own model with single wordlists to get related documents
\end{itemize}

\subsection{Results}

Here's the output of what I got:
\begin{verbatim}
Total Topics: 500
Total Named Topics: 281
Total UnNamed Topics: 219
Total Empty Topics: 80
Total UnNamed Non-empty Topics: 139

Wordlists associated with n topics:
interoperability 1
accuracy 3
usability 4
modularity 5
understandability 6
efficiency 8
security 8
integrity 9
performance 58
verifiability 78
traceability 95
testability 121
correctness 182

Now wordlists that were correlated with each other
correctness, interoperability 1
efficiency, performance 1
efficiency, security 1
efficiency, testability 1
efficiency, verifiability 1
integrity, traceability 1
integrity, verifiability 1
interoperability, interoperability 1
modularity, performance 1
performance, security 1
performance, usability 1
traceability, understandability 1
traceability, usability 1
understandability, verifiability 1
efficiency, traceability 2
security, testability 2
testability, usability 2
accuracy, accuracy 3
accuracy, correctness 3
accuracy, testability 3
accuracy, understandability 3
integrity, testability 3
correctness, integrity 4
security, verifiability 4
usability, usability 4
correctness, efficiency 5
correctness, modularity 5
correctness, security 5
correctness, understandability 5
modularity, modularity 5
security, traceability 5
testability, understandability 5
understandability, understandability 6
efficiency, efficiency 8
security, security 8
integrity, integrity 9
performance, traceability 10
performance, verifiability 12
performance, testability 17
correctness, performance 24
testability, verifiability 26
testability, traceability 32
performance, performance 58
correctness, testability 62
correctness, verifiability 76
traceability, verifiability 76

\end{verbatim}

\section{Exp2: Word Lists extracted from ISO9126 Taxonomy}

We extracted the commit log comments from MaxDB 7.500 and per month
extracted 20 topics using LDA. There are 25 months, thus a total of
topic 500 possible topics, but there were a couple of months with no
commits.

Make a word list (a file) of related words and a label.  These words
were extracted from the ISO9126 Taxonomy.  The main concepts used are:
functionality, reliability, maintainability, usability, efficiency,
and portability.

We determined if a word list was similar to a topic if they shared 1
or more words in common.

\subsection{Observations}

There are many more unnamed topics than before. Only reliability shows up very often.

The words that reliability finds are:
\begin{verbatim}
error 279
bug 25
fails 2
crash 14
failure 1
\end{verbatim}

Usability words
\begin{verbatim}
ui 16
user 8
default 4
gui 3
feature 2
\end{verbatim}

Portability words
\begin{verbatim}
documentation 5
migration 1
\end{verbatim}

Effeciency words
\begin{verbatim}
optimize 18
optimization 4
fast 3
slow 1
\end{verbatim}

And ``dependency'' for maintability

\subsection{Results}


\begin{verbatim}
Total Topics: 500
Total Named Topics: 125
Total UnNamed Topics: 375
Total Empty Topics: 80
Total UnNamed Non-empty Topics: 295

maintainability 1
efficiency 4
portability 5
usability 13
reliability 107

efficiency, reliability 1
maintainability, maintainability 1
maintainability, reliability 1
portability, reliability 1
reliability, usability 2
efficiency, efficiency 4
portability, portability 5
usability, usability 13
reliability, reliability 107
\end{verbatim}


\section{Exp3: Word Lists extracted from ISO9126 Taxonomy combined with Wordnet}

We extended Experiment 2 to include the wordnet words.

\subsection{Observations}

Many more results that before, most topics are named now.


For the efficiency concept:
\begin{verbatim}
add 109
fix 102
base 24
optimize 18
fixing 14
correction 12
command 7
process 5
frame 5
string 4
optimization 4
fast 3
end 3
count 3
upgrade 2
speed 2
solve 2
record 2
avoid 2
slow 1
shorten 1
scratch 1
resolve 1
quick 1
loading 1
frequency 1
estimate 1
corrections 1
control 1
constant 1
administration 1
\end{verbatim}

For functionality:
\begin{verbatim}
fix 102
update 43
window 22
tack 9
key 9
support 8
memory 8
difference 8
mark 7
system 6
ease 6
lease 5
grate 5
frame 5
ring 4
strategy 3
label 3
source 2
power 2
gain 2
buffer 2
buff 2
bar 2
tool 1
machine 1
grant 1
extract 1
complex 1
catch 1
\end{verbatim}

For maintainability
\begin{verbatim}
file 140
log 19
difference 8
mark 7
ease 6
point 5
document 5
environment 4
order 3
improvement 3
end 3
action 3
verify 2
record 2
power 2
post 2
place 2
mode 2
union 1
tract 1
tension 1
sleep 1
representation 1
processing 1
ownership 1
need 1
increase 1
enlargement 1
dependency 1
control 1
condition 1
\end{verbatim}

For portability:
\begin{verbatim}
os 17
support 8
form 8
program 6
function 6
documentation 5
transaction 3
action 3
upgrade 2
routine 1
migration 1
integration 1
\end{verbatim} 

For reliability
\begin{verbatim}
error 279
check 196
change 67
move 60
bug 25
correct 24
age 20
set 18
make 18
ram 17
out 17
ace 15
crash 14
lock 10
sing 9
run 9
list 9
mat 8
form 8
case 8
fail 7
vent 6
stall 6
din 6
point 5
grate 5
get 5
format 5
cover 5
work 4
ting 4
ring 4
fault 4
default 4
adjust 4
turn 3
rap 3
pass 3
order 3
let 3
gel 3
end 3
adapt 3
action 3
static 2
start 2
resume 2
refresh 2
power 2
part 2
modify 2
ill 2
flow 2
find 2
fails 2
drop 2
cut 2
yield 1
worker 1
value 1
union 1
stance 1
sink 1
scratch 1
representation 1
report 1
regard 1
prim 1
pop 1
ping 1
ownership 1
nod 1
instance 1
header 1
grow 1
go 1
failure 1
exit 1
enlargement 1
eld 1
dress 1
develop 1
dependency 1
cycle 1
corrections 1
condition 1
close 1
catch 1
calm 1
administration 1
\end{verbatim}

For usability
\begin{verbatim}
error 279
check 196
test 125
fix 102
window 22
monitor 19
set 18
ram 17
ui 16
view 15
port 15
back 15
elect 12
trace 11
user 8
use 8
man 8
compiler 8
case 8
miss 7
name 6
module 6
function 6
longer 5
etch 5
cover 5
diagnose 4
default 4
gui 3
end 3
aries 3
application 3
wrapper 2
wrap 2
wall 2
top 2
stop 2
possibility 2
match 2
liver 2
libra 2
feature 2
cut 2
cap 2
bar 2
worker 1
ward 1
sort 1
show 1
self 1
resolve 1
relative 1
relation 1
purpose 1
protection 1
present 1
patch 1
owner 1
machine 1
killer 1
go 1
gem 1
extract 1
equal 1
block 1
binary 1
\end{verbatim}


\subsection{Results}
\begin{verbatim}
Total Topics: 500
Total Named Topics: 328
Total UnNamed Topics: 172
Total Empty Topics: 80
Total UnNamed Non-empty Topics: 92

maintainability 44
portability 45
functionality 94
efficiency 200
reliability 225
usability 265

maintainability, portability 5
functionality, maintainability 15
portability, reliability 29
efficiency, maintainability 31
maintainability, reliability 34
maintainability, usability 35
efficiency, portability 38
functionality, portability 39
portability, usability 40
maintainability, maintainability 44
portability, portability 45
functionality, reliability 58
efficiency, functionality 73
functionality, usability 85
functionality, functionality 94
efficiency, reliability 137
efficiency, usability 150
reliability, usability 190
efficiency, efficiency 200
reliability, reliability 225
usability, usability 265
\end{verbatim}



\end{document}

%%% Local Variables: 
%%% mode: latex
%%% TeX-master: t
%%% End: 

@INPROCEEDINGS{5072519,
title={Towards an Ontology for Software Product Quality Attributes},
author={Kayed, A. and Hirzalla, N. and Samhan, A.A. and Alfayoumi, M.},
booktitle={Internet and Web Applications and Services, 2009. ICIW '09. Fourth International Conference on},
year={2009},
month={May},
volume={},
number={},
pages={200-204},
abstract={Recently, quality assurance concept has been developed increasingly to be included in many of our life existing fields; financial, industrial, trading, computing, etc. Software quality product attributes (SWQAs) have been created as a matter of applying the QA concept on the results of Web or desktop application development process, to fit the products with the organizational and global market standards and goals, and to provide it with a competitive advantage value. Web application or software product quality is composed of many attributes such as portability, usability, reliability, modularity. During the recent years, many researchers discussed and presented software attributes in their works which showed that till now there is a lack of consensus on the semantic of many of concepts and terminologies used in this field. Our work is focusing on studying software product quality attributes concepts and terminologies. We conduct several experiments to extract the main concepts for SWQAs. The results show that there is a number of concepts that are frequently used to describe these attributes. Summarizing and formalizing the semantic of the attributes into these concepts presents a common understanding and agreement on the semantic of SWQPAs which can be used by software engineers, researchers, practitioners, and stakeholders.},
keywords={quality assurance, software portability, software quality, software reliability, standardsWeb application development process, competitive advantage value, desktop application development process, global market standards, modularity, organizational market standards, portability, quality assurance, reliability, software product quality attributes, usability},
doi={10.1109/ICIW.2009.36},
ISSN={}, }

